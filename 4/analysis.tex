\section{ANALYSIS} % (fold)
\label{sec:analysis}

\subsection{Recall} % (fold)
\label{sub:recall}
The recall is expected to be almost the same for each iteration. This is due to the fact that the same search keywords are used for all three iterations. This explains why the recall rate is sixty-nine percent for both iterations involving negation.  The recall rate was lower in the original iteration because of the lack of negation, this led to cases where the number of positive hits negated the number of negative hits.  When the sample size is increased dramatically - as in real usage situations where the sample size is in the billions - the recall rate will be roughly constant for the three iterations.  As such, the recall is not considered when comparing the three iterations.  However, the recall rate can be improved using a larger keyword dictionary as a larger keyword dictionary will result in more hits.  This can be achieved either manually, or iterating through an electronic dictionary and using available sentiment keywords to process the definitions of each new entry.
% subsection accuracy (end)

\subsection{Accuracy} % (fold)
\label{sub:Accuracy}
The accuracy level is shown to increase through the iterations - an expected result.  Since the later iterations account for more cases, they are generally more accurate.  This is reflected in the results, where the last iteration is two percent more accurate than the first iteration.  The first iteration only looks for the sentiment keywords without regard for the context with which the keywords are used, while the second and third iterations account for correct and incorrect grammatical usage of negation words that have an inverse effect on the intended sentiment.  In a much larger sample, the difference in accuracy level between the first iteration and the last two iterations should increase dramatically.  The difference in accuracy between the last two iterations should remain close since they both account for negations and double negations are inherently rare since they are emphasized as grammatically incorrect in daily usage.  Double negations will have a bigger effect on younger user bases, where grammar is less ingrained into their writing habits.
% subsection precision (end)

\newpage
\subsection{Runtime} % (fold)
\label{sub:runtime}
The runtime of the first iteration and the last two iterations differ by a mere tenth of a millisecond.  This means that if only runtime were considered, the first iterations should be used in cases where speed is a high priority and the last iteration should be used in situations where speed is a low priority.  As the second and last iterations have a runtime percent difference of 0.241\%, the difference between using either are negligible in terms of runtime.  Thus, for small sample sets - less than one million samples - it should not matter which iteration is used.  This is because ten microseconds per sample will not save a significant amount of time.  However, in large data sets - upwards of one billion - the first iteration would complete over two hours faster than the third iterations, a significant difference when runtime is a high priority.
% subsection runtime (end)

\subsection{Use Cases} % (fold)
\label{sub:use_cases}
While the second iteration is more accurate than the first and faster in runtime than the third, it accomplishes neither the best accuracy nor the best runtime.  Thus from the analyses above, it can be determined that the second iteration should never be used. Instead only two use cases will be presented: when the goal is speed and when the goal is accuracy.  When speed is an issue, but accuracy is not of the upmost importance, the first iteration should be used. An example of such a scenario is a systems health alarm.  Sentiment can be used to quickly identify when certain products are broken by assessing when users complain about the product.  For this use case, the sentiment would have to be updated every few minutes, but the accuracy is not an issue as the alarm does not have to catch all complains, just detect when a certain threshold has been exceeded.  When accuracy is a high priority, the third iteration should be used. An example of such a case is a reaction meter for upcoming products.  Sentiment can be used to accurately identify which areas of the new product users have issues with and changes can be made accordingly. This scenario requires highly accurate results to reflect the actual sentiment of the users. In addition, this use case will not be run constantly - only when new products emerge - thus time is not an issue and hours can be allocated to achieving high accuracy.
% subsection use_cases (end)

\newpage
\subsection{Future Development} % (fold)
\label{sub:future_development}
This report has put forth the simplest method of determining sentiment to demonstrate the power and utility of sentiment analysis on development and user experience.  There are a number of future steps to be taken to increase both accuracy and recall of the sentiment analysis system.  Currently, there is no consideration for context or grammar.  It is assumed that if a message mentions a given keyword (i.e. Work Report) and sentiment keywords are located in the message, then the sentiment keywords represent the user's sentiment towards the given keyword.  This presents a major shortcoming as accuracy will decrease significantly as messages grow longer and contain multiple thoughts and feelings.  This means that the next logical step is to include grammar and context into the analysis algorithm. This is a much more complex problem involving a knowledge of Naive Semantics\cite{dahlgreen}, parsing the message down into its grammatical elements \cite{rayner} and running the same sentiment analysis on the elements.  This allows for the analyzer to know exactly that the sentiment was directed at.  For example, in the sentence ``The boy and the girl loves apples,'' it can be determined that the user has a positive sentiment towards apples and not boy or girl.  This distinction would not be present in the current implementation.
% subsection future_development (end)
% section analysis (end)