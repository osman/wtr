\section*{Contributions} % (fold)
\label{sec:contributions}
\addcontentsline{toc}{section}{Contributions}

I worked as a Data Scientist Intern for the Data Science Team of the Engineering division of Facebook Inc.  I reported directly to Roddy Lindsay - my manager responsible for supervising my day to day work - and was assigned two major projects and a number of various smaller tasks to complete over the course of my four month co-op placement.  The team I worked with consisted of nine people, including myself and another intern. Over the course of the term, I also worked with others in my team as well as providing help with filling data requests from other teams as I became more familiar with the database system.\newline

The main goal of the Data Science team is to design and develop a number of technologies for the organization, utilization and visualization of various data sets.  As Facebook grows and develops into a mature product, so does its already massive database of information.  This includes the interests, likes, hates and conversations of millions of users.  It is the Data Scientist's job to make sense of this data, create tools for accurately and usefully visualizing the data and finally ensuring that other teams understand and utilize the data in a meaningful fashion.  When new products are about to introduced, the company assesses the impact of that new product on the users. This is done via A/B tests, where two similar contained groups are chosen and one exposed to the change. Results are tabulated after a few weeks to see the effects the change would have on site usage overall. Lastly, all teams require data and someone to simplify the statistics required to analyze the data. Thus - in addition to the team's own objectives - a Data Scientist's role is to also fill data requests, perform A/B tests and help other teams understand and analyze relevant data.\newline

Initially, I was assigned a single warm-up project to get acquainted with the technologies and the setup of the company. As I gained understanding of the systems at Facebook, I was given more and more responsibilities.  After completing my warmup project, I was given a two projects to complete over the remaining time of my coop term.  On a daily basis, counts of of total actions and unique users making those actions are aggregated into a table hidden in the database.  This information is useful in filling data requests, as many different teams require that information to assess receptiveness and growth in new products. The initial task was to create a front-end to visualize that hidden data so that those teams can see the data for themselves rather than sending requests to the data team.  I was then given two major projects, the first was from the User Experience team, who wanted to know how users feel about certain products using wall posts and comments on Facebook itself.  I was to build a complete system for pulling all relevant messages, quantifying sentiment and publishing the results to a reporting tool which any team in the company can refer to. This was an important project in moving user experience forward because it allows the company to assess the impact of changes to the product without surveying the users. The second major project involved creating a back-end reporting system for a number of different metrics for the redesigned `pages' product.  These metrics included a number of demographic breakdowns for fans and interactions.  This was accomplished by developing a new operator for Databee, the pipelining tool developed by the Data Science team for Hive. This new operator allowed for demographic breakdowns of anything in Hive such as new user counts, etc...\newline

This report is a direct result of the research and pipelining work on Hive that I was responsible for over the duration of the last four months.  As mentioned earlier, my work consisted of determining a user's sentiment about certain topics based on what they write or how they reply to different user interactions. Traditional methods of determining user sentiment include surveying users and waiting for users to complain, high friction actions compared to the new proposed method. Throughout the term, I had the opportunity to apply my academic knowledge acquired over previous terms, specifically ECE250: Data Structures and Algorithms and ECE251: Programming Languages and Translators.  Lastly this report allows to practice my skills in technical writing and serves as a record of my work completed during my work term on the Data Science team at Facebook.\newline

In the broader scheme of things, this report details and analyses a number of methods that determine a sentiment level through a stored set of messages in a database.  Many companies, including Facebook and their clients, would greatly benefit from this data since they would then be able to determine what their users do and don't like from the second they discuss it on their product.  This is much more efficient and less invasive than the traditional method of surveying users and waiting for users to complain through a complaints system.  The traditional methods are very high friction, requiring the user to care enough about the product to contact the company.  This new method, allows all users to be leveraged into a full-time critic of the product, without disturbing their daily lives or use of the product. This saves the company time and money when making decisions about user experience or figuring out how the user will react to different changes to the product.
% section contributions (end)