\section*{GLOSSARY} % (fold)
\label{sec:glossary}
\addcontentsline{toc}{section}{Glossary}
	\begin{description}
		
		\item[back-end]
denoting a subordinate processor or program, not directly accessed by the user, which performs a specialized function on behalf of a main processor or software system: a back-end database server

		\item[front-end]
a part of a computer or program that allows access to other parts

        \item[Hadoop]
is a software platform that lets one easily write and run applications that process vast amounts of data. Hadoop implements MapReduce, using the Hadoop Distributed File System (HDFS). MapReduce divides applications into many small blocks of work. HDFS creates multiple replicas of data blocks for reliability, placing them on compute nodes around the cluster. MapReduce can then process the data where it is located. \cite{hadoop_about}

         \item[Hive]
    is a data warehouse infrastructure built on top of Hadoop that provides tools to enable easy data summarization, adhoc querying and analysis of large datasets data stored in Hadoop files. It provides a mechanism to put structure on this data and it also provides a simple query language called QL which is based on SQL and which enables users familiar with SQL to query this data. At the same time, this language also allows traditional map/reduce programmers to be able to plug in their custom mappers and reducers to do more sophisticated analysis which may not be supported by the built in capabilities of the language. \cite{hive_about}
    
		\item[mapper]
is a programming operation where each streamed input has a corresponding streamed output

        \item[recall]
the proportion of the number of relevant documents retrieved from a database in response to an inquiry

		\item[reducer]
is a programming operation where a number of streamed inputs has a single corresponding streamed output
	\end{description}
% section glossary (end)