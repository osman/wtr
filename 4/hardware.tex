\section{HARDWARE} % (fold)
\label{sec:hardware}
This section discusses a number of issues and ways of improving the hardware layer of a kdb+ system and how this affects the entire database system.

\subsection{Random Access Memory} % (fold)
\label{sub:ram}
As with any large in-memory database system, Random Access Memory (RAM) is of the upmost importance.  It determines the amount of data that can be collected at a given time, without rolling that data into the historical database as well as the amount of data that a kdb+ session can hold in memory for ultra fast access.  As RAM of any kind is progressively becoming less expensive over time, it becomes easier and more trivial to obtain a significant amount of RAM.  For such a system, at 16 gigabytes of RAM is a required minimum and 32 gigabytes is recommended.  The more data that is stored in-memory (on RAM as opposed to on the hard-drive) the faster that the front-end system can retrieve data.  This is due to the fact that data stored on RAM has an access time of roughly 62.5ns\cite{ram_speed} while some of the fastest hard drives have access times of roughly 13ms\cite{samsung_hdd}.  That means that RAM is roughly one hundred thousand times faster than hard drives and is the main bottleneck front-end systems where speed is crucial.\newline
% subsection ram (end)

\subsection{Allocation of Swap Space} % (fold)
\label{sub:swap_space}
Swap space is a term given to a dedicated area in the hard drive for swapping excess data to and from RAM.  Although the hard drive is almost one hundred thousand times slower than RAM in retrieving data[see Section \ref{sub:ram}], swap space is still necessary because it is still not possible to always have enough RAM to suit our needs. Low priority data could also be kept in the swap space as there is no need to retrieve it as fast as the high priority data.  This allows for the company to buy less RAM and use extra hard drive space as swap space.  The swap space needs to be at least as large or twice as large as the RAM in this case because there will be times where historical data will need to be analyzed and historical data could be anywhere from one day to many years.  Should there not be enough RAM, there should be at least enough swap space to hold those tables to wait for processing in RAM.\newline

This presents a major bottleneck since a lack of RAM will not cause the system to crash, but a lack of swap space will.  If there is not enough swap space, the system will try to load the data into RAM, then the remaining into swap, and crash if there is not enough.  There is an easy solution to this problem however and that is to just increase the amount of swap space available to two or three times the available RAM and to have more RAM available.  Hard drives have begun to decline in cost tremendously in the last few years and it is a trivial matter to purchase new hard drives solely for the purpose of extra swap space.\newline
% subsection swap_space (end)

\subsection{Hard Drives - Space and Quantity} % (fold)
\label{sub:hard_drives}
Any database, no matter how big or small, requires significant hard drive space.  This is due to the fact that a significant amount of data is being collected and needs to be stored.  As hard drives become less and less expensive as well as having more and more space, companies are beginning to acquire more and more data to analyze.  It then becomes a challenge because there will never be enough hard drive space.  Roughly every six months, new drives will have to be added to accommodate new data being streamed in from various sources.  This is a bottleneck or inefficiency that cannot be solved or fixed, however there are methods of minimizing these effects through the way data is stored and accessed.  That will be discussed in the next section.\newline

Another major consideration, in terms of efficiency and bottlenecks, is always the speed at which data can be retrieved from the hard drives.  There are many methods to speed up data retrieval from the hard drives, one of which is to use a Redundant Array of Independent Disks (RAID) system.  A RAID system not only ensures that the data is backed up and safe, but can also increase the speeds of data retrieval by multitudes.  This is due to the fact that many hard drives are used in parallel, so more data can be retrieved from the set of hard drives in the same given amount of time.  This ensures that the biggest bottleneck of the database system, runs as fast as possible and is as efficient as possible.  It is also important to have a number of seperate RAID systems on the same server so that parallelization techniques in kdb+ can also be fully utilized.  Those techniques will be discussed in the software section of the report.\newline
% subsection hard_drives (end)

\subsection{Partitioned Tables} % (fold)
\label{sub:data_storage}
As the amount of data stored increases, it becomes more and more cumbersome to keep tables as a single file, thus a common method of separating these tables is by partitioning the data in a consistent manner - often by date.  As historical data is being partitioned, each days worth of data resides in their own separate file.  When the database is queried, only the days that are of importance are loaded into memory.  This significantly reduces the amount of data loaded into the limited RAM at a time and significantly speeds up query times.  An effect of this would be to always have the partitioning factor as the first condition when narrowing down a query by a number of criteria.  If the partitioning factor is not first, the query would then load the whole table before narrowing down the search, which significantly lengths query times.  This is a significant bottleneck, because it is not considered as the historical data grows.  Companies - ones in the financial industry in particular - collect and store terabytes and petabytes worth of data over the course of a year and queries begin to take longer and longer if proper measures are not taken.\newline
% subsection data_storage (end)

% section hardware (end)