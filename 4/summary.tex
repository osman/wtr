\section*{Summary} % (fold)
\label{sec:summary}
\addcontentsline{toc}{section}{Summary}

The main purpose of this report is to detail and analyze a quantification method for determining sentiment level in users through sent and received messages.  The analysis will be based on three metrics: recall, accuracy and runtime.  Specific use cases for each iteration as well as future developments will be discussed.\newline
 
This report is intended for an audience with an intermediate knowledge of databases, standard stream architectures present in most modern programming languages and a firm grasp of Python.\newline

The major points covered in this report are three iterations of a method in quantifying sentiment from a written message. The framework and a description of each iteration of the method is provided with an analysis of the results based on three criteria: recall, accuracy and runtime.  Finally, a more complex method of quantifying sentiment is introduced, but not detailed.\newline

The major conclusions in this report are that sentiment is a powerful tool in assessing mass reactions to changes in a product without any direct contact to the users themselves.  This allows for quicker response to new issues and an overall better user experience.\newline

The major recommendations in this report are that time and resources should be spent on increasing the sentiment keyword library, which would allow for an improved rate of recall.  One team member should research more applications of user sentiment in areas such as advertising or suggestion-making based on user preferences.  Lastly, at least one co-op student from an engineering program should be given the project of developing a parsing infrastructure that breaks down the messages by grammar and context, allowing for much more accurate results.
% section summary (end)