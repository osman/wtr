\section*{Contributions} % (fold)
\label{sec:contributions}
\addcontentsline{toc}{section}{Contributions}

I worked as a General Technology Intern for the eTrading Execution Team (eT) of the Fixed Income IT division of Barclays Capital.  I reported directly to Sonia Saldanha - my line manager - and was assigned a number of projects to complete over the course of my four month co-op placement.  The team I worked with consisted of nine people, including myself.  I worked closely with Vivek Viswanathan, who was responsible for supervising my day to day work.  Over the course of the term, I also worked with others in my team as well as providing help with writing database queries as I became more familiar with the database system.\newline

The main goal of the eT Executions team is to design and develop a number of technologies for the first leg of a trade.  A trade typically consists of the trader and a client negotiating a price for a buying or selling a bond, the trade to go through the Trade Capture \& Workflow system (TCW), get priced through the Pricing and Risk system (Rushmore) and finally stored in our databases.  The eT Executions team is responsible for the health and development of the first section, the systems that allow the trader to view current prices, risk and negotiate the most beneficial prices with clients.  There are a number of systems that fall under the eT Executions team, the largest of which is called eT.  eT is a plugin system that displays analytic information and allows for traders to switch on and off the pricing systems as well as the auto-negotiation system. Lastly, they are also responsible for the Internal Brokerage system (IB) which tracks all trades internal to the company.  Both are front-end systems that the companies traders use to complete their daily tasks.  There is an ongoing effort to increase the efficiency and throughput of each of the systems so that the company can cope with any increases in business.  These changes are mostly implemented through re-architecture and upgrading existing systems.\newline

Initially, I was assign a single major project to complete over the course of four months.  As I demonstrated proficiency in my work, I was given more and more responsibilities.  By the end of the four months, I had been assigned four major projects.  The initial task was to design a system for displaying internal reports globally. After much consideration, it was decided that a web portal as a front-end and KDb - a database system - queries on the back-end would be the most efficient solution.  Each type of report (Management, Trader, Sales, On/Off Statistics, etc.) would have to be translated into equivalent Q queries - the querying language for KDb - and the web portal would use those queries to generate dynamic reports for the user.  This posed a difficult design problem since these reports needed to be global, however, London and New York had different KDb instances with differing data tables, making it impossible to use the exact same queries on both.  I was given the task of developing all the back-end work generating the reports and my counterpart in London - another intern - would complete the web portal to display these reports.  As I became more proficient in Q and used KDb more, I realized that certain overnight batch processes were written in such a way that significantly lengthened the process.  After some discussion with my manager, I was decided that these batch processes would be optimized by me since they were also delaying the progress of my own project by making the KDb servers less stable.  This was my second project.  Over the course of a few weeks, I optimized the queries so that they completed their task approximately three hundred times faster.  It then became much easier to complete my original project.  I was then given the task of creating two new reports for the Sales team and to analyze statistics regarding system on and off times. These were completed and plans were made to integrate them into the web portal.  I ended up completing all my tasks ahead of schedule, so my final project was to take over development of the web portal for the New York reports.  The web portal was built off of Python using a werkzurg web-server, a Mako templating system and javascript.  Vigorous testing was performed at each stage of each project to ensure that each of the projects met my own and my supervisor's standards.\newline

This report is a direct result of the research and optimization work on KDb that I was responsible for over the duration of the last four months. As mentioned earlier, my work consisted of creating new Q queries, optimizing old Q queries and analyzing the process with which KDb deals with queries as a system.  KDb and Q - its querying language - have very little documentation and my analyses soon became very valuable to the team and anybody who uses an implementation of the KDb database system.  In addition, these techniques - while very specific to the capabilities of KDb - could also be applied to other database systems such as SQL, Sybase, or Oracle servers.  Throughout the term, I had the opportunity to apply my academic knowledge acquired over previous terms, specifically ECE250: Data Structures and Algorithms. Lastly this report allows me to practice my skills in technical writing and serves as a record of my work completed during my work term at the Fixed Income IT department of Barclays Capital.\newline


In the broader scheme of things, this report analyses how a number of concepts can increase the efficiency and throughput of a database system, such as KDb, and reduce the number of failed queries and server crashes over time.  Many companies, including Barclays Capital, would benefit greatly from this since they rely heavily of such system for services such as automatic trading, computing risk, computing pricing, and storing valuable data.  By following certain principles, less people will be required to support these systems and can be allocated to supporting other systems. Saving the company time and allowing for developers to further innovate in other areas.
% section contributions (end)