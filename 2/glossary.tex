\section*{GLOSSARY} % (fold)
\label{sec:glossary}
\addcontentsline{toc}{section}{Glossary}
	\begin{description}
		
		\item[back-end]
denoting a subordinate processor or program, not directly accessed by the user, which performs a specialized function on behalf of a main processor or software system: a back-end database server

		\item[carriage return]
the action of moving the carriage of a typewriter back to the left margin
		
		\item[Extensible Markup Language (XML)]
a metalanguage that allows users to define their own customized markup languages, especially in order to display documents on the World Wide Web

		\item[front-end]
a part of a computer or program that allows access to other parts
		
		\item[repository]
a place where data or materials are stored

		\item[workflow]
The path and systems used in the linked flow of activities with a specific start and finish that describe a process. The flow defines where inputs are initiated, the location of decision points and the alternatives in output paths, and is used in systems that perform automatic routing.
	\end{description}
% section glossary (end)