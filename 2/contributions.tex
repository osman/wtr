\section*{Contributions} % (fold)
\label{sec:contributions}
\addcontentsline{toc}{section}{Contributions}

I worked as a Technical Systems Analyst (TSA) in the Common Components Centre of Excellence (CCCoE) department of Global Technology and Operations (GTO) at Royal Bank of Canada (RBC).  I reported directly to Juraj Liska - my project manager - and was assigned my own project to complete over the course of my four month co-op placement.  The team I worked with was very small. It consisted of myself and Paul Cashmore, the solutions architect.  Paul Cashmore was responsible for code reviews of my design and negotiating product specifications with the other departments.\newline

The main goal of the CCCoE department is to design and develop a number of technology and channel neutral ``Common Components'' for RBC.  These ``Common Components'' are used by all other departments in GTO to create client-facing front-end software that powers everything from handling transactions to logging daily ATM usage. This method of reusable development holds its greatest benefit in quicker and more efficient integration of new technologies. As long as new revisions to the ``Common Components'' abide by any predetermined specifications, CCCoE is free to revise the components. This allows CCCoE to refactor code in an effort to make it more efficient or robust without needing to make the same changes in many different locations. Any revisions made on the ``Common Components'' are instantly reflected in all the front-end softwares that link to the components. This means that there are less human errors in implementation because the revisions are only implemented once. In addition, more time and resources can be allocated to ensuring the quality and maintenance of the components instead of implementing changes over a wide base of products. CCCoE is also responsible for testing all of its own products and is essentially the backbone of all operations in RBC.\newline

I was assigned a single major project to complete over the course of four months.  My task was to design and develop an automated workflow for consuming data. The data steward  - person responsible for keeping the relevant data updated - would submit data in the form of a Microsoft Excel document or a Comma Separated Value (CSV) document; two very common forms of data storage. The data would be fed into my system, which then transforms it into another common form of data storage for easy manipulation, Extensible Markup Language (XML).  Once transformed into the master document format that I specified, I was able to use an Extensible Stylesheet Language Transformation (XSLT) processor to create any number of different required output files from the master file. These output files are specified by a number of different Extensible Stylesheet Language (XSL) documents and consisted of two major categories: contexts and maps. Contexts are output documents that display a single piece of information, the context, while a maps are output documents that map one context to another. The system was designed to be flexible and able to handle numerous input files and input types. Properties files were used instead of hard-coded properties so that every user is able to customize with personalized user settings. Also, vigorous testing was performed on all steps of the process.\newline

This report is a direct result of the research and work that I was responsible for, and my project was completed over the duration of the last four months. As mentioned earlier, my work consisted of creating an automated workflow, hence my supervisor recommended I try the Continuous Integration method to aid me in developing my automated process. Upon much research of the Continuous Integration methodology, I soon came to the realization that the concepts in the methodology could also be applied to my workflow and increase the efficiency of the workflow itself. This report allows for the demonstration of evaluation and critical analysis of a recommended method in software development and its application in a broader field. Throughout the term, I had the opportunity to apply my academic knowledge acquired over previous terms, specifically ECE250: Data Structures and Algorithms. Lastly this report allows me to practice my skills in technical writing and serves as a record of my work completed during my work term at the GTO department of the Royal Bank of Canada.\newline

In the broader scheme of things, this report analyses hows the concepts entailed in Continuous Integration apply to workflow automation.  Many companies, including RBC, would benefit from the greater efficiency of an automated workflow as problems such as integration and complex manual operations only need to be solved once and then become a non-issue thereafter.  This also eliminates human errors, stemming from the re-entry of data at multiple locations as humans are erased from the workflow equation. By eliminating humans from these previously manual operations, resources are freed and can then be reallocated to more pressing projects or further the level of innovation within the company.
% section contributions (end)