\section{BENEFITS ANALYSIS} % (fold)
\label{sec:benefits_analysis}

\subsection{Automation} % (fold)
\label{sub:automation}
There are numerous reasons that companies would invest time and money on automation of many of their workflows. The most prevalent of these reasons would be the increase in efficiency, decrease in human error, and decrease in the number of resources required to maintain the current production throughput. A computer is able to work much faster than a human is able to, thus automation of complex workflows means that work is completed in a more efficient manner. In addition to this increase efficiency, the exclusion of humans in the workflow inherently significantly decreases the amount of human error involved with the process.  There will no longer be missing files or misspelt words. This increase in efficiency also bring with it a decrease in the number of people required to complete a task as complex workflows that originally required many.

	\begin{table}[ht]
		\centering
		\caption[Increase in output per worker]{Increase in output per worker over a decade\cite{Terborgh}}
		\label{tab:automation_benefits}
		\begin{tabular}{| l | c |} 
			\hline
			Country & Increase in Output per Worker 1952-62 (Percent)\\ \hline
			Japan & 99 \\ \hline 
			Italy & 60 \\ \hline 
			Germany & 50 \\ \hline
			Austria & 47 \\ \hline
			Netherlands & 45 \\ \hline
			France & 40 \\ \hline
			Belgium & 37 \\ \hline
			Norway & 36 \\ \hline
			United States & 21 \\ \hline
			United Kingdom & 20 \\ \hline
			Canada & 19 \\ 
			\hline 
		\end{tabular}
	\end{table}
	
As seen in Table.~\ref{tab:automation_benefits}, ``since the advantages of technological progress are recognized not only officially but generally... not even the technological alarmists question the benefits of of technology; they are concerned with the associated disbenefits.''\cite{Terborgh} The downside of increased efficiency and decreased required resources is the possibility that ``larger numbers of those who become redundant as a result of automation lack skills to fit them for new jobs, and so we find ourselves faced with the paradox that, on the one hand, there are numerous empty jobs, and on the other there are numerous people who remain unemployed, because there is not enough retraining available.''\cite{Bagrit} There is also the important matter of initial cost. Automating workflows require significant initial resources since a whole new infrastructure must be formed to handle the new system. Companies may decide that this initial cost is too large and that they are not able to afford such a major change at their current level. 
% subsection automation (end)

\subsection{Efficiency} % (fold)
\label{sub:efficiency}
Only two of the three key concepts yield an increase in workflow efficiency. These would be maintaining a single source repository and keeping the build fast.  Both these concepts work to allow the overall workflow to finish faster.  First, the single source repository allows the project to always know where to find the parts it need, whether it be a software file or a physical toy part, the repository is the only place the automation is required to look, saving expensive searching or decision time. The major disadvantage to a single source repository, in terms of efficiency, would be that as the number of parts increases the repository becomes very large and you start losing the benefit of a single repository. It may then be beneficiary to split the repository into smaller more specialized repositories to retain the advantage. In terms of efficiency, keeping the build fast using a combination of prioritization and parallelization of tasks yields a much greater efficiency, without any disadvantages in terms of efficiency. Lastly, having a self-testing build may be beneficial to the robustness and quality of the code, however this concept is a major disadvantage to overall efficiency. This is due to the fact that the more tests there are, the better the quality, but the lower the efficiency as testing takes a lot of time.
% subsection efficiency (end)

\subsection{Resources Required} % (fold)
\label{sub:resources_required}
The resources required to implement each of the three concepts discussed in this report are very great and not all companies are able to afford such an expense. This is the greatest disadvantage to applying Continuous Integration concepts to workflow automation. The advantages however, in terms of resources required, are also very great should a company be able to afford the initial costs. Once the automation is in place, very few of the original resources assigned to the manual task would be needed anymore. These valuable assets to the company could then be reassigned to different tasks and that allows the company to further increase its productivity and efficiency. There is always also the option of saving money and not retaining any resources that are no longer required. In either case, the company downsizes on resources - leaving only those that handle maintaining and developing the automated system - and increases efficiency at the same time.  By applying these Continuous Integration concepts, there is the possibility of the company further decreasing the resources required, due to the further increases in efficiency.
% subsection resources_required (end)

% section benefits_analysis (end)